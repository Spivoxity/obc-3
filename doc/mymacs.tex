% jmsmac.tex 
% $Id: jmsmac.tex 3 2006-12-31 20:35:33Z mike $

% Macros for Mike's book format

%\input jmsdef
% jmsdef.tex 
% $Id: jmsdef.tex 2 2006-06-12 09:26:54Z mike $

% Basic definitions for JMS format (also used in MetaPost)

\catcode`@=11

\let\protect=\relax

%
% Font loading
%

% Six font schemes:
%
%   Scheme	Body font	\iflucida	CM fonts
%   0 (default)	lbr		true
%   1		lsr		true
%   2		cmr		false		sauter
%   3		cmss		false		sauter
%   4		cmr		false		basic
%   5		cmss		false		basic
%
% Schemes 4 and 5 are like 2 and 3, but use just the basic set of CM fonts.

\newif\iflucida
\newif\ifbasiccm
\newcount\fontscheme

% Say `\def\fonts{2}\input book' to get CMR
\ifx\fonts\undefined\else\fontscheme=\fonts\fi

\ifnum\fontscheme<2 \lucidatrue\else\lucidafalse\fi
\ifnum\fontscheme>3 \basiccmtrue \advance\fontscheme by-2 \else\basiccmfalse\fi

% To deal with the Sans font scheme (\fontscheme=1), we distinguish
% between the body font \tenrm (lbr or lsr as appropriate) and the
% math font \tenmrm (always lbr).  Same (and more important) for \tenit
% vs \tenmti and \tenbf vs \tenmbf.
%
% So there are three potentially different italic fonts: the italic
% used for running text (\tenit), the text italic font that is used for
% maths (\tenmti), and the math italic font (\teni).  There are two
% font-changing commands: 
%
%     Command	Horiz mode	Math mode
%
%     \it	\tenit		\tenmti
%     \mti	\tenmti		(no effect)
% 
% In math mode, you get \teni by default with no font-changing command.

% Font substitution for basic CM

\def\fontsubst#1#2{\expandafter\def\csname subst@#1\endcsname{#2}}

\ifbasiccm
  \fontsubst{cmcsc8}{cmcsc10 at8pt}
  \fontsubst{cmcsci8}{cmcsc10 at8pt}
  \fontsubst{cmssbx8}{cmssbx10 at8pt}
  \fontsubst{cmitt8}{cmitt10 at8pt}

  \fontsubst{cmcsc9}{cmcsc10 at9pt}
  \fontsubst{cmcsci9}{cmcsc10 at9pt}
  \fontsubst{cmssbx9}{cmssbx10 at9pt}
  \fontsubst{cmitt9}{cmitt10 at9pt}

  \fontsubst{cmcsci10}{cmcsc10}

  \fontsubst{cmcsci12}{cmcsc10 scaled\magstep1}
  \fontsubst{cmitt12}{cmitt10 scaled\magstep1}
  \fontsubst{cmsy12}{cmsy10 scaled\magstep1}
  \fontsubst{cmcsc12}{cmcsc10 scaled\magstep1}
  \fontsubst{cmex12}{cmex10 scaled\magstep1}
  \fontsubst{cmssbx12}{cmssbx10 scaled\magstep1}

  \fontsubst{cmr14}{cmr10 scaled\magstep2}
  \fontsubst{cmss14}{cmss10 scaled\magstep2}

  \fontsubst{cmr20}{cmr10 scaled\magstep4}
  \fontsubst{cmss20}{cmss10 scaled\magstep4}
  \fontsubst{cmssi20}{cmssi10 scaled\magstep4}
  \fontsubst{cmssbx20}{cmssbx10 scaled\magstep4}

  \fontsubst{cmss24}{cmss10 scaled\magstep5}
  \fontsubst{cmssbx24}{cmssbx10 scaled\magstep5}
\fi

% Use e.g. \@font\twelverm{cmcsci10 scaled\magstep1} to load a font,
% but with a possible substitution cmcsci10 --> cmcsc10, giving the
% effect of \font\twelverm=cmcsc10 scaled\magstep1
\def\@font#1#2{
  \expandafter\let\expandafter\next
    \csname subst@\expandafter\killscale#2 scaled\killscale\endcsname
  \ifx\next\relax \font#1=#2 
  \else \expandafter\getscale#2scaledscaled\getscale 
    \font#1=\next\scale \fi}

\def\killscale#1 scaled#2\killscale{#1}
\def\getscale#1scaled#2scaled#3\getscale{\def\scale{#2}%
  \ifx\scale\empty\else \def\scale{ scaled #2}\fi}


%
% Size selection
%

\newdimen\fontquad 
\newdimen\fontbody
\newdimen\blockindent

\newfam\sffam
\newfam\arfam % used for arrow font in Lucida

\newif\ifleaded
\iflucida\leadedtrue\fi

\newtoks\sizetoks

% Can set running text in 8, 9, 10, 11 or 12 point

\def\setsize#1#2#3#4#5#6{%
% #1 = text size
% #2 = script size
% #3 = scriptscript size
% #4 = quad and body size (CM)
% #5 = body size (Lucida)
% #6 = jot
  \ifleaded \fontbody=#5 \else \fontbody=#4 \fi
  \fontquad=#4 \jot=#6
  \parindent=\fontquad
  \blockindent=2\fontquad
  \def\@textsize{#1}\def\@scriptsize{#2}\def\@scriptscriptsize{#3}%
  	% NB: these sizes are used by the math quotes "..."
  \setfamily0{mrm} \setmface0{rm} \setface{mrm}
  \setfamily1{i}
  \setfamily2{sy}
  \setexfamily
  % The remaining families are not to be used in subscripts
  \smallfamily\itfam{mti} \setmface\itfam{it} \setface{mti} 
  \smallfamily\bffam{mbf} \setmface\bffam{bf} \setface{mbf} 
  \smallfamily\sffam{sf}  \setmface\sffam{sf} 
  \iflucida\smallfamily\arfam{ar}\fi
  \the\sizetoks
  \normalbaselines\rm}

\def\setface#1{\expandafter\edef\csname p#1\endcsname{%
  \expandafter\noexpand\csname\@textsize#1\endcsname}\ignorespaces}
\def\setmface#1#2{\expandafter\edef\csname p#2\endcsname{%
  \fam#1\expandafter\noexpand\csname\@textsize#2\endcsname}\ignorespaces}
\def\setfamily#1#2{%
  \textfont#1=\csname\@textsize#2\endcsname
  \scriptfont#1=\csname\@scriptsize#2\endcsname
  \scriptscriptfont#1=\csname\@scriptscriptsize#2\endcsname\ignorespaces}
\def\smallfamily#1#2{%
  \textfont#1=\csname\@textsize#2\endcsname\ignorespaces}
\def\setexfamily{%
  \textfont3=\csname\@textsize ex\endcsname
  \scriptfont3=\csname\@textsize ex\endcsname
  \scriptscriptfont3=\csname\@textsize ex\endcsname}

\def\eightpoint{\setsize{eight}{six}{five}{10pt}{11pt}{2pt}}
\def\ninepoint{\setsize{nine}{six}{five}{11pt}{12pt}{2.5pt}}
\def\tenpoint{\setsize{ten}{seven}{five}{12pt}{13pt}{3pt}}
\def\elevenpoint{\setsize{eleven}{eight}{six}{13pt}{14pt}{3.5pt}}
\def\twelvepoint{\setsize{twelve}{eight}{six}{14pt}{15pt}{3.5pt}}

\def\deffont#1{%
  % Add definition of face to \sizetoks
  \expandafter\sizetoks\expandafter{\the\sizetoks\setface{#1}}
  % Define the font-changing command
  \expandafter\edef\csname #1\endcsname{\protect\expandafter
    \noexpand\csname p#1\endcsname}}

\def\rm{\protect\prm} \def\mrm{\protect\pmrm} \def\it{\protect\pit}
\def\mti{\protect\pmti} \def\bf{\protect\pbf} \def\mbf{\protect\pmbf}
\def\sf{\protect\psf}

\deffont{sl} \deffont{sc} \deffont{sci} \deffont{sfb} \deffont{vsf}
\deffont{vsfi} \deffont{tt} \deffont{tti} \deffont{kr} \deffont{ki}

\def\normalbaselines{%
  \lineskip=\normallineskip
  \baselineskip=\fontbody
  \lineskiplimit=\normallineskiplimit
  \bigskipamount=\fontbody plus5pt minus3pt
  \medskipamount=0.5\fontbody plus4pt minus2pt
  \abovedisplayskip=\medskipamount
  \belowdisplayskip=\medskipamount
  \abovedisplayshortskip=\abovedisplayskip
  \belowdisplayshortskip=\belowdisplayskip
  \setbox\strutbox=\hbox{\vrule height0.7\fontbody depth0.3\fontbody width0pt}}

% We do not have oldstyle digits.
\let\oldstyle=\undefined

% \sl is not used for math setting
\textfont\slfam=\nullfont \scriptfont\slfam=\nullfont
  \scriptscriptfont\slfam=\nullfont


%
% Loading of fonts
%

\def\smallsizes{
  \do{five}{5.2}{5}{5}
  \do{six}{6.1}{6}{6}
  \do{seven}{6.9}{7}{7}}

\def\bigsizes{
  \do{eight}{7.8}{8}{8}
  \do{nine}{8.6}{9}{9}
  \do{ten}{9.5}{10}{10}
  \do{eleven}{10.4}{11}{10 scaled\magstephalf}
  \do{twelve}{11.2}{12}{12}}

\def\allsizes{\smallsizes\bigsizes}

\def\loadfont#1#2#3#4#5{
  \def\do##1##2##3##4{% name lusize cmsize basiccmsize
    \edef\f@nt{\ifcase\fontscheme #2 at##2pt\or #3 at##2pt\or
      \ifbasiccm #4##4\else #4##3\fi\or \ifbasiccm #5##4\else #5##3\fi\fi}
    \expandafter\@font\csname ##1#1\endcsname\f@nt}}

\def\specialfont#1#2#3{%
  \ifcase\fontscheme \@font#1{#2} \or\@font#1{#2} 
    \or\@font#1{#3} \or\@font#1{#3}\fi}

% Extra sizes for Lucida
\def\lfourteen{13pt}
\def\ltwenty{18pt}
\def\ltwentyfour{22pt}

\loadfont{mrm}  {lbr}    {lbr}    {cmr}    {cmr}    \allsizes
\loadfont{i}    {lbmi}   {lbmi}   {cmmi}   {cmmi}   \allsizes
\loadfont{sy}   {lbms}   {lbms}   {cmsy}   {cmsy}   \allsizes
\iflucida
  \loadfont{ar} {lbma}   {lbma}   {***}    {***}    \allsizes
\fi

\loadfont{rm}   {lbr}    {lsr}    {cmr}    {cmss}   \bigsizes
\loadfont{it}   {lbi}	 {lsi}    {cmti}   {cmssi}  \bigsizes
\loadfont{ex}   {lbme}   {lbme}   {cmex}   {cmex}   \bigsizes
\loadfont{bf}   {lbd}    {lsd}    {cmbx}   {cmssbx} \bigsizes
\loadfont{sf}   {lsr}    {lsr}    {cmss}   {cmss}   \bigsizes
\loadfont{sfi}	{lsi}	 {lsi}	  {cmssi}  {cmssi}  \bigsizes
\loadfont{mti}	{lbi}    {lbi}    {cmti}   {cmti}   \bigsizes
\loadfont{mbf}	{lbd}    {lbd}    {cmbx}   {cmbx}   \bigsizes
\loadfont{sl}	{lbsl}   {lso}    {cmsl}   {cmssi}  \bigsizes
\loadfont{sc}	{lbrsc}  {lbrsc}  {cmcsc}  {cmcsc}  \bigsizes
\loadfont{sci}	{lbosc}  {lbosc}  {cmcsci} {cmcsci} \bigsizes
\loadfont{sfb}	{lsd}    {lsd}    {cmssbx} {cmssbx} \bigsizes
\loadfont{vsf}	{vlsr}   {lbtn}   {cmtt}   {cmtt}   \bigsizes
\loadfont{vsfi}	{vlsi}   {lbtno}  {cmitt}  {cmitt}  \bigsizes
\loadfont{tt}	{lbtn}   {lbtn}   {cmtt}   {cmtt}   \bigsizes
\loadfont{tti}	{lbtno}  {lbtno}  {cmitt}  {cmitt}  \bigsizes
\loadfont{kr}   {lbkr}   {lbkr}   {cmss}   {cmss}   \bigsizes
\loadfont{ki}   {lbki}   {lbki}   {cmssi}  {cmssi}  \bigsizes

% Set the skewchars
\def\do#1#2#3#4{
  \expandafter\skewchar\csname #1i\endcsname='177
  \expandafter\skewchar\csname #1sy\endcsname='60
}
\allsizes

% Special fonts

% Fourteen point
\specialfont\fourteenrm{lbr at\lfourteen}{cmr14}
\specialfont\fourteensf{lsr at\lfourteen}{cmss14}

% Twenty point
\specialfont\twentyrm{lbr at\ltwenty}{cmr20}
\specialfont\twentysf{lsr at\ltwenty}{cmss20}
\specialfont\twentysfb{lsb at\ltwenty}{cmssbx20}
\specialfont\twentysfi{lsi at\ltwenty}{cmssi20}

% Twenty-four point
\specialfont\twentyfoursf{lsr at\ltwentyfour}{cmss24}
\specialfont\twentyfoursfb{lsd at\ltwentyfour}{cmssbx24}
\specialfont\twentyfourfunky{lbki at\ltwentyfour}{cmssbx24}

% Typeface loads a single face in multiple sizes.  The Lucida-adjusted
% size is used in each case, even if the font scheme is CM.
% Best used to load funky variants of Lucida
\def\typeface#1#2{%
  \def\do##1##2##3##4{% name lusize cmsize basiccmsize
    \expandafter\@font\csname ##1#1\endcsname{#2 at##2pt}}%
  \bigsizes
  \deffont{#1}}


%
% Adjustments for Lucida
%

\iflucida

\def\neq{\mathrel{\fam\arfam\mathchar"7094}}

% Adjusted for LucidaNewMath-Extension at 9.5pt and math axis at 3.13pt
% Note: delimiter increments are 5.5pt (as opposed to 6pt in CM)
% These work in \tenpoint size only.

\def\big#1{{\hbox{$\left#1\vbox to8.20\p@{}\right.\n@space$}}}
\def\Big#1{{\hbox{$\left#1\vbox to10.80\p@{}\right.\n@space$}}}
\def\bigg#1{{\hbox{$\left#1\vbox to13.42\p@{}\right.\n@space$}}}
\def\Bigg#1{{\hbox{$\left#1\vbox to16.03\p@{}\right.\n@space$}}}
\def\biggg#1{{\hbox{$\left#1\vbox to17.72\p@{}\right.\n@space$}}}
\def\Biggg#1{{\hbox{$\left#1\vbox to21.25\p@{}\right.\n@space$}}}
\def\n@space{\nulldelimiterspace\z@ \m@th}

% define some extra large sizes - always done using extensible parts

\def\bigggl{\mathopen\biggg}
\def\bigggr{\mathclose\biggg}
\def\Bigggl{\mathopen\Biggg}
\def\Bigggr{\mathclose\Biggg}

%% %  Following is needed if the roman text font is NOT just LucidaBright
%% 
%% %  Draw the small sizes of `[' and `]' from LBMO instead of LBR
%% 
%% \mathcode`\[="4186 \delcode`\[="186302
%% \mathcode`\]="5187 \delcode`\]="187303
%% 
%% %  Draw the small sizes of `(' and `)' from LBMO instead of LBR
%% 
%% \mathcode`\(="4184 \delcode`\(="184300
%% \mathcode`\)="5185 \delcode`\)="185301
%% 
%% %  The small sizes of `{' and `}' are already drawn from LBMS instead of LBR
%% 
%% %  Draw small `/' from LBMO instead of LBR
%% 
%% \mathcode`\/="013D \delcode`\/="13D30E
%% 
%% %  Draw  `=' and `+' from LBMS instead of LBR
%% 
%% \mathcode`\=="3283 \mathcode`\+="2282
%% 
%% % May want to comment out this last one if text font IS known to be LBR

% plain.tex draws upper case upright Greek from roman text font ---
% --- need to draw instead from LucidaNewMath-Extension

\mathchardef\Gamma="03D0
\mathchardef\Delta="03D1
\mathchardef\Theta="03D2
\mathchardef\Lambda="03D3
\mathchardef\Xi="03D4
\mathchardef\Pi="03D5
\mathchardef\Sigma="03D6
\mathchardef\Upsilon="03D7
\mathchardef\Phi="03D8
\mathchardef\Psi="03D9
\mathchardef\Omega="03DA

% The dot accent has moved
\def\dot{\mathaccent"7005 }

\def\TeX{T\kern-.18em\lower.4ex\hbox{E}\kern-.1emX}

\def\copyright{{\rm\char"A9}}

\def\matrix#1{\null\,\vcenter{\normalbaselines\m@th
    \ialign{\hfil$##$\hfil&&\quad\hfil$##$\hfil\crcr
      \mathstrut\crcr\noalign{\kern-0.9\baselineskip}
      #1\crcr\mathstrut\crcr\noalign{\kern-0.9\baselineskip}}}\,}

% \p@renwd is used in \bordermatrix as the width of big extensible left paren
% i.e. where rows and columns are labelled outside parenthesized matrix
% \newdimen\p@renwd
\setbox\z@=\hbox{\tenex B} \p@renwd=\wd\z@ % width of the big left (

% following changed because `(' is not large enough for strut in LBMO
\def\mathstrut{\vphantom{f}}

% In n-th root, don't want the `n' to come too close to the radical
\def\r@@t#1#2{\setbox\z@\hbox{$\m@th#1\sqrt{#2}$}
  \dimen@\ht\z@ \advance\dimen@-\dp\z@
  \mkern5mu\raise.6\dimen@\copy\rootbox \mkern-7.5mu \box\z@}

\fi


%
% Math stuff
%

\def~{\relax\ifmmode\,\else\penalty10000\ \fi}
\def\{{\relax\ifmmode\lbrace\else\iflucida\char`\{\else$\lbrace$\fi\fi}
\def\}{\relax\ifmmode\rbrace\else\iflucida\char`\}\else$\rbrace$\fi\fi}

% ! is an Ord, not a Close as in plain TeX
\mathcode`\!="0021

% Math spacing is a bit more chunky that in plain TeX
\thinmuskip=4mu plus2mu minus1mu
\medmuskip=5mu plus2mu minus2mu
\thickmuskip=6mu plus3mu

% In cmr10 and plain TeX, an ordinary space is 12/36 plus 6/36 minus 4/36 em,
% 1 mu is 2/36 em, and the three mu skips are
%	thin  3 mu 			= 6/36 em
%	med   4 plus 2 minus 4 mu	= 8/36 plus 4/36 minus 8/36 em
%	thick 5 plus 5 mu		= 10/36 plus 10/36 em

% @ is a binary operator
\mathcode`\@="2040

% "stuff" sets stuff with math text italic.  We use an \hbox to deal
% with \_ nicely, and use \mathselect to get the appropriate size:
\def\kwote#1"{\mathselect\kw@te{#1}}
\def\kw@te#1#2{\hbox{\csname #1mti\endcsname #2\/}}
\mathcode`\"="8000
{\catcode`\"=\active \global\let"=\kwote}

% Like mathpalette, but the argument is a font size
\def\mathselect#1#2{\mathchoice{#1\@textsize{#2}}{#1\@textsize{#2}}%
  {#1\@scriptsize{#2}}{#1\@scriptscriptsize{#2}}}

\def\isc#1{\hbox{\sci \@cap#1\@cap\/}} % constant in caps and small caps
\def\@cap#1#2\@cap{#1\lowercase{#2}}

%\def\star{\mathord{*}}

\catcode`@=12

\tenpoint


\catcode`@=11

\newif\ifpdf \pdffalse
\ifx\pdfoutput\undefined \else
  \ifnum\pdfoutput>0 \pdftrue \fi
\fi

%
% Page layout
%

\newdimen\grid 			
\newdimen\normaltopskip 	
\newdimen\pagewidth 		
\newdimen\textwidth
\newdimen\pageheight 		
\newdimen\colsep 		

\ifleaded \grid=13pt \else \grid=12pt \fi
\normaltopskip=\grid
\pagewidth=350pt                
\textwidth=\pagewidth
\colsep=24pt                    

\def\setpage#1#2{
  \pageheight=\normaltopskip 
  \advance\pageheight by\ifleaded #1\else #2\fi\grid
  \@setpage}

\def\@setpage{
  \hsize=\textwidth
  \vsize=\pageheight
  \topskip=\normaltopskip
  \figlimit=\pageheight
  \dimen\topins=\figlimit
  \dimen\botins=\figlimit}

\def\normalpage{\setpage{39}{43}}
\def\stretchpage{\setpage{50}{55}}

\def\setheight#1{
  \pageheight=#1
  \divide\pageheight by\grid
  \multiply\pageheight by\grid
  \advance\pageheight by-4\grid % allow for head- and footlines
  \@setpage}


%
% Spacing commands
%

\def\medspace{\vspace\medskipamount}
\def\bigspace{\vspace\bigskipamount}

\def\bigskipneg{\vskip-\bigskipamount} % to eliminate the `shoulder pad effect'

\def\enspace{\kern0.5\fontquad\relax}
\def\quad{\hskip\fontquad\relax}
\def\qquad{\hskip2\fontquad\relax}

\def\vspace{\par\begingroup\afterassignment\@vspace\skip0=}
\def\@vspace{\ifdim\lastskip<\skip0 \removelastskip\vskip\skip0\fi\endgroup}

\newif\ifinfloat \infloatfalse
\newif\ifnomargin \nomarginfalse

\def\nomargin{\nomargintrue}


%
% Headlines and footlines
%

\newif\iftitle \titlefalse
\newif\ifblank \blankfalse

\newtoks\foottoks
\newif\ifonefoot \onefootfalse

\def\onefoot#1{\onefoottrue\foottoks={#1}}

\def\rcsid$#1${\onefoot{{\tensf\$#1\$}}}

\headline={\ifblank\hfil\else\iftitle\hfil\else
  \ifodd\pageno\rightheadline\else\leftheadline\fi\fi\fi}

% Note: no page number appears in the footer of a single-page document
% that begins with a chapter heading.
\footline={%
  \ifblank \hfil 
  \else\iftitle \ifodd\pageno\rightnumfoot\else\leftnumfoot\fi 
  \else \theonefoot \fi\fi}

\def\theonefoot{\ifonefoot \the\foottoks\hfil \global\onefootfalse \fi}

% Use \hfill here so \hfil\break can be used to split headings
\def\leftheadline{\tenit\thepageno\quad\lhead\hfill}
\def\rightheadline{\hfill\tenit\rhead\quad\thepageno}

\def\leftnumfoot{\tenit\thepageno\hfil\theonefoot}
\def\rightnumfoot{\theonefoot\hfil\tenit\thepageno}

\def\lhead{} % Will be chapter title
\def\rhead{\firstmark} % Will be 'A' head at page top

\def\thepageno{\folio}


%
% Footnotes (borrowed parts from plain and manmac)
%

\skip\footins=\grid plus0.5\grid
\dimen\footins=20\grid
\count\footins=1000

\def\footnote{\fnmark\fntext}

\def\footnum{$^{\number\footcnt}$}

\def\fnmark{\edef\@sf{\spacefactor\the\spacefactor}%
  \global\advance\footcnt by1
  \hyperanchor@\fntag{fn}%
  \hyperlink{\fntag a}{\footnum}\@sf}

\def\fntext{\insert\footins\bgroup\eightpoint
  \interlinepenalty=\interfootnotelinepenalty
  \let\par=\endgraf
  \leftskip=0pt \rightskip=0pt
  \splittopskip=\ht\strutbox % top baseline for broken footnotes
  \splitmaxdepth=\dp\strutbox \floatingpenalty=20000
  \noindent 
  \hyperanchor{\fntag a}\hyperlink\fntag{\footnum}\enspace
  \bgroup\strut
  \aftergroup\@foot\let\next}
\def\@foot{\strut\egroup}


%
% Output routine (borrows parts of plain and manmac)
% 

\newinsert\margin
\dimen\margin=\maxdimen \count\margin=0 \skip\margin=0pt

% Preamble stuff to include PostScript at start of DVI file.
% This is superior to using a simple \special, because it does not
% add to the main vertical list, so that an initial \clearpage remains
% a no-op.
\newbox\preambox \setbox\preambox=\vbox{}
\def\preamble#1{\global\setbox\preambox=\vbox{\unvbox\preambox#1}}
\def\dopreamble{\unvbox\preambox \global\let\dopreamble=\relax}

\output={\outputpage\pagebody \resetpage}

% JMS 8/10/01: moved insertion stuff to \pagebody so footnotes appear
% below \botins.  This affects
% insertions with double columns -- but I don't think they worked anyway.

\def\outputpage#1{\offinterlineskip
  \shipout\vbox{
    \dopreamble
    \makeheadline
    \vbox to\pageheight{\boxmaxdepth=\maxdepth #1}
    \makefootline}
  \advancepageno
  \ifvoid255
    \ifnum\outputpenalty>-20000 \else\dosupereject\fi
  \else
    \unvbox255 \penalty\outputpenalty
  \fi}

\def\makeheadline{%
  \vbox to2\grid{\vss
    \hbox to\pagewidth{\the\headline}
    \kern-\prevdepth \vskip\grid}}

\def\makefootline{%
  \normalbaselines\baselineskip=2\grid 
  \hbox to\pagewidth{\the\footline}}

\def\pagebody{%
  \ifvoid\margin\else \marginalnotes \fi
  \ifvoid\topins\else \unboxinsertions\topins \fi
  \dimen@=\dp255 % with ragged bottom, depth may fall into page foot
  \unvbox255
  \ifvoid\botins\else \unboxinsertions\botins \fi
  \ifvoid\footins\else
    \vskip\skip\footins
    \footnoterule
    \unvbox\footins\fi
  \ifr@ggedbottom \kern-\dimen@ \vfil \fi}

\def\footnoterule{\kern-3pt
  \hrule width 2in \kern 2.6pt} % the \hrule is .4pt high

\def\marginalnotes{\hbox{\hskip\pagewidth\hskip 12pt
    \vbox to0pt{\kern6pt \box\margin \vss}}}

\def\resetpage{\global\titlefalse \global\blankfalse
  \global\dimen\topins=\figlimit}

\def\frontmatter{\par\vfil\break\pageno=-1 }
\def\body{\par\vfil\break\pageno=1 }

\def\raggedbottom{\topskip=\normaltopskip
  \advance\topskip by0pt plus2\grid
  \r@ggedbottomtrue}
\def\flushbottom{\topskip=\normaltopskip \r@ggedbottomfalse}

\def\clearpage{\par
  \ifdim\prevdepth>-1000pt \dimen0=\prevdepth    % Allow depth of last
    \ifdim\dimen0>\maxdepth \dimen0=\maxdepth\fi % box to fall into page foot
    \kern-\dimen0 \fi
  \vfil\supereject}


%
% Double columns
%

% Borrowed from the TeXbook (Appendix E, page 417) ...

\newbox\partialpage

\def\begindoublecolumns{\begingroup
  \output={\global\setbox\partialpage=\vbox{\unvbox255\bigskip}}\eject
  \output={\doublecolumnout}
  \hsize=\pagewidth \advance\hsize by-\colsep \divide\hsize by2
  \vsize=\pageheight \multiply\vsize by2 \advance\vsize by\baselineskip}
\def\enddoublecolumns{\output={\balancecolumns}\eject
  \endgroup \pagegoal=\vsize}

\def\doublecolumnout{\splittopskip=\topskip \splitmaxdepth=\maxdepth
  \dimen@=\pageheight \advance\dimen@ by-\ht\partialpage
  \setbox0=\vsplit255 to\dimen@ \setbox2=\vsplit255 to\dimen@
  \outputpage\pagesofar \resetpage
  \unvbox255 \penalty\outputpenalty}

\def\pagesofar{\unvbox\partialpage \nointerlineskip
  \wd0=\hsize \wd2=\hsize 
  \hbox to\pagewidth{\box0\hfil\box2}}

\def\balancecolumns{\setbox0=\vbox{\unvbox255} \dimen@=\ht0
  \advance\dimen@ by\topskip \advance\dimen@ by-\baselineskip
  \divide\dimen@ by2 \splittopskip=\topskip
  {\vbadness=10000 \loop \global\setbox3=\copy0
    \global\setbox1=\vsplit3 to\dimen@
    \ifdim\ht3>\dimen@ \global\advance\dimen@ by1pt \repeat}
  \setbox0=\vbox to\dimen@{\unvbox1}
  \setbox2=\vbox to\dimen@{\unvbox3}
  \pagesofar}

% ... end of borrowing

% For smaller pieces of two-column work, here's code that makes a
% two-column box:

\def\twocolbox{\setbox0=\vbox\bgroup
  \advance\hsize by-\colsep \divide\hsize by2
  \prevdepth=\dp\strutbox % Simulate \topskip=\ht\strutbox
  \aftergroup\twosplit\let\next}

\def\twosplit{{\splittopskip=\ht\strutbox \vbadness=10000
  \dimen0=\ht0 
  \advance\dimen0 by3\baselineskip % Right column should be a little short
  \divide\dimen0 by2
  \setbox1=\vsplit0 to\dimen0
  \hbox{\valign{##\vfil\cr\unvbox1\cr\noalign{\colrule}\unvbox0\cr}}}}

\def\colrule{\hskip\colsep}


%
% Floating insertions
%

\newif\ifpagefig
\newif\ifbotfig

\newdimen\figlimit
\newinsert\botins
\count\topins=1000 \skip\topins=0pt
\count\botins=1000 \skip\botins=0pt

\def\insertion{\begingroup\setbox0=\vbox\bgroup
  \infloattrue\prevdepth=0pt\lineskip=0pt
  \hyperanchor@\instag{ins}}
\def\endins{\egroup
  \ifdim\ht0>0.7\pageheight \pagefigtrue\fi
  \insert\ifbotfig\botins\else\topins\fi{%
    \penalty100
    \splittopskip=0pt \splitmaxdepth=\maxdimen \floatingpenalty=0
    \ifpagefig 
      \dimen@=\dp0 
      \vbox to\pageheight{\vfil \unvbox0 \kern-\dimen@ \vfil}
    \else
      % Increase height by 1.5\grid, then round up to next mult of \grid:
      \dimen@=\ht0 \advance\dimen@ by2.5\grid
      \divide\dimen@ by\grid \multiply\dimen@ by\grid
      \ifbotfig
        \vbox to\dimen@{\vfil \unvbox0}
      \else
        \vbox to\dimen@{\unvbox0 \vfil}
      \fi
    \fi}%
  \endgroup
  \pagefigfalse \botfigfalse}

% Unbox the insertions to equalize the spaces between them, but box
% the result to the desired height.
\def\unboxinsertions#1{\vbox to\ht#1{\unvbox#1 \unboxthem}}
\def\unboxthem{\setbox0=\lastbox \ifvbox0{\unboxthem}\unvbox0\fi}

\def\figure{\insertion \def\docaption##1{##1\figcnt{Figure}}}
\def\botfig{\botfigtrue \insertion \def\docaption##1{##1\figcnt{Figure}}}
\def\pagefig{\pagefigtrue \insertion \def\docaption##1{##1\figcnt{Figure}}}
\def\table{\insertion \def\docaption##1{##1\tabcnt{Table}}}
\let\endfig=\endins
\let\endtab=\endins

\def\caption{\docaption\makecaption}
\def\contcaption{\docaption\makecontcap}

\def\makecaption#1#2#3{\bigskip
  \global\advance#1 by1 \xdef\lastlabel{{\thechapdot\number#1}{\instag}}
  \hbox to\pagewidth{\hfil\ninepoint 
    \bf #2 \thechapdot\number#1: \it #3\hfil}}

\def\makecontcap#1#2#3{\bigskip
  \xdef\lastlabel{{\thechapdot\number#1}{\instag}}
  \hbox to\pagewidth{\hfil\ninepoint 
    \bf #2 \thechapdot\number#1 \rm (cont.): \it #3\hfil}}

\input epsf

%\def\epsfsize#1#2{0.5#1}

\def\graphic#1{\placepic{\hss\image{#1}\hss}}
\def\twingraphics#1#2{\placepic{\hfil\image{#1}\hfil\image{#2}\hfil}}

\def\placepic#1{\medskip\hbox to\pagewidth{#1}\medskip}

\def\image#1{%
  \openin\testin=#1
  \ifeof\testin \closein\testin \placeholder{#1}%
  \else \closein\testin \epsfbox{#1}\fi}

\def\placeholder#1{\immediate\write16{Missing figure #1}%
    \vbox to2in{\hrule\vfill
      \hbox to3in{\hfill[Insert graphic {\vsf #1} here]\hfill}
      \vfill\hrule}}


%
% Chapters and sections
%

\newcount\chap
\newcount\sec		
\newcount\subsec
\newcount\figcnt
\newcount\tabcnt
\newcount\footcnt
\newcount\exno

\newif\ifnobreak % This is a hack (stolen from LaTeX) to prevent
		 % \label from allowing bad page breaks.

\newtoks\chapreset
\newtoks\secreset

\chapreset={\global\sec=0 \global\figcnt=0 
  \global\tabcnt=0 \global\footcnt=0 \global\exno=0 }
\secreset={\global\subsec=0 }

\def\addtoks#1#2{\expandafter\global\expandafter#1\expandafter{\the#1#2}}

%%%% Chapters

\def\chapheading#1#2#3{% number title author
  \global\titletrue \dimen\topins=0pt % No headlines or floats this page
  \vbox to15\grid{
    \tenpoint
    \hbox{\vbox to2\grid{}\twelvepoint\sf #1}
    \hbox to\pagewidth{\strut\leaders\hrule height6pt depth-4.5pt\hfil}
    \baselineskip=2\grid
    \pretolerance=10000 \rightskip=0pt plus1fil
    \def\\{\hfil\break}
    {\twentysf\def\pit{\twentysfi}\noindent #2\par}
    \baselineskip=1.5\grid
    \vskip2\grid
    {\twelvepoint\sf \noindent \ignorespaces #3\par}
    \vfill}
  \global\nobreaktrue \global\everypar={\afterheadpar}}

\newif\ifapp \appfalse
\def\appendix{\apptrue\chap=0\addtoc\tocinsert{}{Appendix}{}{}}
\def\appendices{\apptrue\chap=0\addtoc\tocinsert{}{Appendices}{}{}}

\def\chapstart#1#2#3{% numcode title author
  \clearpage\resetpar
  #1
  \the\chapreset 
  \xdef\lastlabel{{\thechap}{chap.\thechap}}
  \bookmark1{chap.\thechap}{\thechap\space #2}
  \chapheading{\ifapp Appendix\else Chapter\fi\ \thechap}{#2}{#3}\nobreak
  \xdef\lhead{#2} \addmark{\thechap\ \ #2}
  \ifapp \addtoc\toclineii\thechap{#2}\thepageno\bmark
  \else \addtoc\toclinei\thechap{#2}\thepageno\bmark \fi}

\def\chapbegin{\chapstart{\advance\chap by1}}

\def\nonumchap#1#2{%
  \clearpage\resetpar
  \the\chapreset 
  \bookmark@1{nonumchap}{#1}
  \chapheading{}{#1}{#2}\nobreak
  \xdef\lhead{#1} \addmark{#1}}

\outer\def\preface{\nonumchap{Preface}{}
  \addtoc\toclinei{}{Preface}\thepageno\bmark}

\def\thechap{\ifapp \Alph\chap \else \number\chap \fi}
\def\thesec{\thechapdot\number\sec}
\def\thesubsec{\thesec.\number\subsec}

% The space after 0 is needed, or TeX inserts \relax for some reason
\def\thechapdot{\ifnum\chap=0 \else\thechap.\fi}

\def\Alph#1{\ifcase#1 \or A\or B\or C\or D\or E\or F\or G\or H\or I
  \or J\or K\or L\or M\or N\or O\or P\or Q\or R\or S\or T\or U\or V
  \or W\or X\or Y\or Z\fi}


%%%% Sections

\def\beforesechead#1{%
  % An expanded version of \vspace 2\fontbody
  {\skip0=\lastskip
    \vskip 0pt plus6\fontbody \penalty50 \vskip 0pt plus-6\fontbody
    \ifdim\skip0<#1
      \ifdim\skip0=0pt \else\vskip-\skip0\fi
      \vskip #1 \fi}}

\def\aftersechead#1{\nobreak\vskip#1\nobreaktrue
  \global\everypar={\afterheadpar}}

\def\sechead#1{\hbox{\twelvepoint\sfb #1}}
\def\subsechead#1{\hbox{\tenpoint\sfb #1}}

\def\secbegin#1{%
  \par \resetpar \advance\sec by1 \the\secreset
  \xdef\lastlabel{{\thesec}{sec.\thesec}}
  \beforesechead{2\fontbody}
  \sechead{\thesec \quad #1}
  \bookmark2{sec.\thesec}{\thesec\space \killparens{#1}}
  \addmark{\thesec\ \ #1}
  \ifapp\else \addtoc\toclineii{\thesec}{#1}\thepageno\bmark \fi
  \aftersechead\fontbody}

\def\nonumsec#1{%
  \par \resetpar 
  \beforesechead{2\fontbody}
  \sechead{#1}
  \bookmark@2{nonumsec}{#1}
  \aftersechead\fontbody}

\def\subsecbegin#1{%
  \par \resetpar \advance\subsec by1
  \xdef\lastlabel{{\thesubsec}{subsec.\thesubsec}}
  \beforesechead\fontbody
  \subsechead{\thesubsec \quad #1}
  \bookmark3{subsec.\thesubsec}{\thesubsec\space \killparens{#1}}
  \aftersechead{0pt}}

\def\nonumsubsec#1{%
  \par \resetpar 
  \beforesechead\fontbody
  \subsechead{#1}
  \bookmark@3{nonumsubsec}{#1}
  \aftersechead{0pt}}

\def\afterheadpar{\resetpar\hskip-\parindent}
\def\addmark#1{{\enableprotect \mark{\killparens{#1}}}}

% Fix up the mark when a section head is
%   \section Lists (page~\pageref{s:lists}).
% It's used for (optional) parts of labs too.
\def\killparens#1{\killp@rens#1 ()\killp@rens}
\def\killp@rens#1 (#2)#3\killp@rens{#1}

% Chapter headings can have an ``author'' specified after a slash
% Mostly used for single-chapter papers
\def\sl@sh#1/#2/#3\sl@sh{{#1}{#2}}

\outer\def\chapter{\ifstar{\chapstar@.}\chap@@@}
\outer\def\chapstar#1.{\chapstar@@{#1}}
\def\chap@@@#1.{\expandafter\chapbegin\sl@sh#1//\sl@sh}
\def\chapstar@. #1.{\chapstar@@{#1}}
\def\chapstar@@#1{\expandafter\nonumchap\sl@sh#1//\sl@sh
  \addtoc\toclinei{}{#1}\thepageno\bmark}

\outer\def\chapnum{\afterassignment\chapnum@\count255=}
\def\chapnum@#1.{\expandafter\chapnum@@\sl@sh#1//\sl@sh}
\def\chapnum@@{\chapstart{\chap=\count255}}

\outer\def\section{\ifstar{\secstar@.}\sec@@@}
\def\sec@@@#1.{\secbegin{#1}}
\def\secstar@. #1.{\nonumsec{#1}}
\outer\def\secstar#1.{\nonumsec{#1}}

\outer\def\subsection{\ifstar{\subsecstar@.}\subsec@@@}
\def\subsec@@@#1.{\subsecbegin{#1}}
\def\subsecstar@. #1.{\nonumsubsec{#1}}
\outer\def\subsecstar#1.{\nonumsubsec{#1}}

\outer\def\summary{\nonumsec{Summary}}
\outer\def\exercises{\nonumsec{Exercises}}
\outer\def\endexercises{}

\newcount\labno \labno=0
\outer\def\labchap#1.{\advance\labno by1
  \chapbegin{Lab \spellout\labno: #1}{}
  \xdef\lastlabel{{\the\labno}{\cdr\lastlabel}}}

\def\spellout#1{\ifcase #1\of zero\or one\or two\or three\or 
	four\or five\or six\or seven\or eight\or nine\or 
	ten\else \error{can't count beyond ten}\fi}


%
% Verbatim listing
%

{\obeyspaces\global\let =\ }

\def\uncatcodespecials{\def\do##1{\catcode`##1=12 }\dospecials}

\newif\ifblanks \blanksfalse
\newdimen\verbindent

% This can be changed to use e.g. typewriter type for verbatim
\def\verbfont{\vsf}
\def\verbifont{\vsfi}
\def\progfont{\vsf}

\def\setupverb{\uncatcodespecials \verbfont
% \catcode`\|=\active \verbvbar % not needed if using vfont
% \catcode`\`=\active % not needed if using vfont
  }

\def\setupverbatim{\resetpar \parskip=0pt \setupverb
  % If setting in normal measure, use full page width instead
  \ifdim\hsize=\textwidth \hsize=\pagewidth\fi
  \parindent=0pt
  \ifnomargin\global\nomarginfalse\else\advance\parindent by\blockindent\fi
  \let\par=\verbpar
  \verbindent=0pt \verbspaces 
  \everypar{\hskip\verbindent \global\verbindent=0pt}% 
  \obeylines \obeyspaces}

\def\verbpar{\ifblanks\leavevmode\endgraf
  \else\ifvmode\medskip\else\endgraf\fi\fi}

{\obeyspaces\gdef\verbspaces{\def {\verbsp}}}
\def\verbsp{\ifvmode\advance\verbindent by0.5em \else\nobreak\ \fi}

{\catcode`\`=\active \gdef`{\relax\lq}}

{\catcode`\|=\active \gdef\verbvbar{\def|{\hbox to0.5em{\hfil\vertbar\hfil}}}}
\chardef\vertbar=`\|

\def\verbatim{\beforedisplay\begingroup\setupverbatim\doverbatim}
\def\ttverbatim{\beforedisplay\begingroup\setupverbatim\tt
  \let\verbsp=\ \doverbatim}
\def\endverb{\endgroup\afterdisplay\ignorespaces}

\def\program{\beforedisplay\begingroup\setupverbatim\progfont\doprogram}
\def\suprogram{\beforedisplay\begingroup\setupverbatim\progfont%
	\superprog\doprogram}
\def\escprogram{\beforedisplay\begingroup\setupverbatim\progfont%
	\escprog\doescprog}
\let\endprog=\endverb

\def\superprog{\catcode`\^=\active \catcode`\%=\active \@super}
{\catcode`\^=\active \catcode`\%=\active
  \global\def\@super{\def^##1^{\@@super{##1}}\def%##1%{\hgrey{##1}}}}
\def\@@super#1{$^{#1}$}

\def\escprog{\catcode`\\=0 }
{\obeylines \gdef\doescprog^^M#1\endprog{#1\endprog}}

% Listing with line numbers
\def\listing{\beforedisplay\begingroup
  \setupverbatim\progfont\numlines\dolisting}
\def\endlisting{\endgroup\afterdisplay\ignorespaces}

\newcount\lineno
\def\lnum{\global\advance\lineno by1 
  \llap{{\eightrm\number\lineno}\enspace}}

\def\numlines{\lineno=0 \blankstrue
  \everypar={\lnum \hskip\verbindent \verbindent=0pt }}

% Include a file and list it verbatim
\def\verbfile#1{\beforedisplay\begingroup
  \setupverbatim\input#1 \endgroup\afterdisplay}

% Ditto, but with line numbers
\def\listfile#1{\beforedisplay\begingroup
  \setupverbatim\numlines\input#1 \endgroup\afterdisplay}

% Interaction -- enclose input in _underscores_
\def\interact{\beforedisplay\begingroup
  \setupverbatim\catcode`\_=\active\italverb\dointeract}
\let\endinter=\endverb

{\catcode`\_=\active 
  \gdef\italverb{\def_##1_{\def\next{##1}%
      \ifx\next\empty \underscore\else{\verbifont ##1}\fi}}}
\chardef\underscore=`\_

{\catcode`\|=0 \catcode`\\=12 |obeylines%
  |gdef|doverbatim^^M#1\endverb{#1|endverb}%
  |gdef|doprogram^^M#1\endprog{#1|endprog}%
  |gdef|dolisting^^M#1\endlist{#1|endlisting}%
  |gdef|dointeract^^M#1\endinter{#1|endinter}}

\def\verb{\begingroup\setupverb\obeyspaces\doverb}
\def\doverb#1{\def\next##1#1{##1\endgroup}\next}

\def\menu{\beforedisplay\begingroup\displ@y
  \halign\bgroup\hskip\leftskip\hskip\blockindent##\hfil&&\quad##\hfil\cr}
\def\endmenu{\crcr\egroup\endgroup\afterdisplay}


%
% Exercises
%

\newif\ifansflag \ansflagfalse
\newtoks\exlabs \exlabs={}

\def\exlabel#1{\par\resetpar\bigspace\noindent
  \hyperanchor{\cdr\lastlabel}%
  \ifhyperex\hyperlink\anslink{\bf \car\lastlabel#1}%
  \else{\bf \car\lastlabel#1}\fi\the\exlabs}

\def\theexnum{\thechapdot\number\exno}

\def\exercise{\ansflagfalse \advance\exno by1 \exlabs={}
  \xdef\lastlabel{{\theexnum}{ex.\theexnum}}%
  \xdef\anslink{ans.\theexnum}\meltenum}

\def\savedanswer#1{\def\lastlabel{{#1}{ans.#1}}%
  \def\anslink{ex.#1}\meltenum}

% Melt a following \enumerate into the exercise label
% Or a following \alphenum or \romenum
% Also a following \footnote ... but not both!
\def\meltenum{\futurenonspacelet\next\melti} 
\def\melti{\ifx\next\enumerate \def\next{\meltii\enumitem}%
  \else\ifx\next\alphenum \def\next{\meltii\alphitem}%
  \else\ifx\next\romenum \def\next{\meltii\romitem}%
  \else\ifx\next\footnote \let\next=\meltiii
  \else\ifx\next\label \let\next=\meltiv
  \else \let\next=\meltdefault \fi\fi\fi\fi\fi \next}
\def\meltii#1#2{\exlabel\relax\enspace
  \begingroup\listsetup\let\item=#1\let\listitem=\meltitem}
\def\meltiii\footnote{\exlabel\fnmark\quad\fntext}
\def\meltiv\label#1{\exlabs={\label{#1}}\meltenum}
\def\meltdefault{\exlabel{}\quad}

\def\endex{\par \ifansflag\else\immediate\write16
    {Exercise \theexnum \space has no answer}\fi}

\newwrite\ansout
\newif\ifansopen \ansopenfalse
\def\checkans{\ifansopen\else
  \immediate\openout\ansout=\jobname.ans\global\ansopentrue\fi}

% see p.422 of the TeXbook
\long\def\answer{\ansflagtrue\endex\checkans
  \immediate\write\ansout{\string\savedanswer{\theexnum}}
  \begingroup\uncatcodespecials\obeylines\copyanswer}
\long\def\anstext{\checkans
  \begingroup\uncatcodespecials\obeylines\copyanstext}

{\obeylines%
  \gdef\copyanswer#1^^M{\let\qend=\qendex\copyloop}%
  \gdef\copyanstext#1^^M{\let\qend=\qendans\copyloop}%
  \gdef\copyloop#1^^M{\def\next{#1}%
    \ifx\next\qend \let\next=\endgroup 
    \else \immediate\write\ansout{\next} \let\next=\copyloop\fi
    \next}}

{\catcode`\|=0 \catcode`\\=12 
  |gdef|qendex{\endex}%
  |gdef|qendans{\endans}}

% If the answers are to appear in a different size of type, that is
% the responsibility of the MS file.

\def\theanswers{\let\exercise=\undefined
  \immediate\closeout\ansout 
  \input\jobname.ans\relax}


%
% Displays
%

% Displayed equations are ranged left on 2em
% Stolen from Appendix D, p. 375--6
\everydisplay={\displaysetup}
\def\displaysetup#1$${\displaytest#1\eqno\eqno\displaytest}
\def\displaytest#1\eqno#2\eqno#3\displaytest{\leftdisplay{#1}{#2}}

\def\leftdisplay#1#2{\hbox to\displaywidth{%
    \hskip\leftskip\hskip\displayindent\hskip\blockindent
    $\displaystyle#1$\hfil \if!#2!\else$#2$\fi}$$}

\def\beforedisplay{\par\ifnobreak\global\nobreakfalse
  \else\ifinfloat\else\penalty\predisplaypenalty
    \vskip\abovedisplayskip\fi\fi}
\def\afterdisplay{\ifinfloat\else\penalty\postdisplaypenalty
  {\parskip=\belowdisplayskip\noindent}\fi}

% \beginlistings ... \endlistings for whole sections that consist of
% one unindented display.
\def\beginlistings{\begingroup\infloattrue\blockindent=0pt}
\def\endlistings{\endgroup}

% \display is used for misc. alignments
\def\display{\beforedisplay\begingroup\displ@y}
\def\enddisplay{\endgroup\afterdisplay}

% This is the same as the \displ@y in plain.tex
\def\displ@y{\global\dt@ptrue\openup\jot\m@th
  \everycr{\noalign{\ifdt@p \global\dt@pfalse 
        \ifdim\prevdepth>-1000pt
          \vskip-\lineskiplimit \vskip\normallineskiplimit \fi
      \else \penalty\interdisplaylinepenalty \fi}}}
\def\@lign{\tabskip\z@skip\everycr{}} % restore inside \displ@y

% \reopen is used to fix the opening up if a \noalign happens
\def\reopen{\global\dt@ptrue}

% \displaylines for multi-line displayed equations
\def\displaylines{\display
  \halign to\hsize\bgroup
    \hskip\leftskip\hskip\blockindent$\@lign##$\hfil\tabskip=0pt plus1fil
      &$\@lign##$\tabskip=0pt\cr}
\def\enddisplines{\crcr\egroup\enddisplay}
    
% Here's \aligneqns to replace \eqalign(no)
\def\aligneqns{\beforedisplay\begingroup\displ@y
  \halign to\hsize\bgroup
    \hskip\leftskip\hskip\blockindent\hfil$##$
     &${}##$\hfil\tabskip=0pt plus1fil\cr}
\def\endalign{\crcr\egroup\endgroup\afterdisplay}

% \syntax for display of grammars and derivations
% Text is set in \tt, but with the spacing of \it (changed for Lucida)
%	\u{expr} gives expr underlined
%	_foo_ gives foo set in \it
\def\syntax{\display\syn
  \def\also{\noalign{\medskip}} 
  \halign to\hsize\bgroup
    \hskip\leftskip\hskip\blockindent##\hfil&\hfil\quad##\quad\hfil
	&##\hfil \tabskip=0pt plus1fil\cr}
\def\endsyntax{\crcr\egroup\enddisplay}

\def\syn{%  
  \def\arrow{$\rightarrow$}\def\orr{$\mid$}\def\Arrow{$\Rightarrow$}%
  \def\u##1{$\underline{\smash{\hbox{##1}}}$}%
  \def\{{$\lbrace$}\def\}{$\rbrace$}\def\[{$[$}\def\]{$]$}%
  \def\({$($}\def\){$)$}%
  \catcode`\_=\active \setupsynbar \verbfont}

\def\itspacing{{\it\global\skip1=\fontdimen2\font plus\fontdimen3\font
    minus\fontdimen4\font \global\skip2=\fontdimen7\font}%
  \spaceskip=\skip1 \xspaceskip=\skip1 \advance\xspaceskip by\skip2}

{\catcode`\_=\active 
  \global\def\setupsynbar{\let_=\synbar}
  \global\def\synbar#1_{{\it #1\/}}}

\def\envtable{\beforedisplay\nointerlineskip
  \hbox\bgroup\hskip\blockindent
    \vbox\bgroup\offinterlineskip
      \def\lead{height2pt&\omit&&\omit&\cr}
      \def\hline{\lead\noalign{\hrule}\lead}
      \hrule
      \halign\bgroup\vrule##&\strut\quad##\hfil\quad&\vrule##&
        \quad##\hfil\quad&\vrule##\cr
    \lead}
\def\endenvtable{\crcr
    \lead\egroup
  \hrule\egroup\egroup\nointerlineskip\afterdisplay}


%
% PS tricks
%

\def\grey{0.9}

\ifpdf
  \def\pdf@colour{0 g 0 G} % Should be global variable

  \def\@grey#1{%
    \pdfliteral{\grey\space g \grey\space G}#1\pdfliteral{\pdf@colour}}
\else
  \def\@grey#1{%
    \special{ps:gsave \grey\space setgray}#1\special{ps:grestore}}
\fi

% \hgrey -- \hbox with grey background
\def\hgrey{\setbox0=\hbox\bgroup\aftergroup\hgreyi\let\next=}
\def\hgreyi{%
  \setbox0=\hbox{\thinspace\unhbox0\thinspace}%
  \setbox1=\hbox{\hbox to\wd0{\strut\@grey{\leaders\vrule\hfill}}}%
  \ht1=0pt \dp1=0pt \wd1=0pt 
  \leavevmode \hbox{\box1\box0}}

% \vgrey -- \vbox with grey background
\def\vgrey{\setbox0=\vbox\bgroup\aftergroup\vgreyi\let\next=}
\def\vgreyi{%
  \setbox1=\vbox to0pt{\hbox to\hsize{\negthinspace
    \@grey{\leaders\hrule height\ht0 depth\dp0\hfill}\negthinspace}
   \vss}
  \vbox{\box1\nointerlineskip\box0}}


%
% ML (and Oberon!) code
%

\def\mlprog{\beforedisplay\begingroup
  \codemargin=\leftskip \lineno=0
  \ifnomargin\global\nomarginfalse\else\advance\codemargin by\blockindent\fi}
\def\endml{\endgroup\afterdisplay}

\let\obprog=\mlprog
\let\endob=\endml

\newdimen\yycolumn \yycolumn=180pt

\def\yyprog{\mlprog
  \def\LI{\penalty100\tabalign&}%
  \def\NL{\cr}%
  \s@tt@b\hskip\codemargin&\hskip\yycolumn&\cr}
\let\endyy=\endml

\def\lcomment{\hbox to1.5em{$(*$\hfil}}
\def\lcommentstar{\hbox to2.5em{$(*{*}$\hfil}}
\def\rcomment{\hbox to1.5em{\hfil$*)$}}

\newdimen\codemargin
\newif\ifnumlines \numlinesfalse
\newskip\rcommentskip \rcommentskip=0pt plus1fil
\def\rcfrac{0.5}
\def\slab{1.0em}

\def\LI{\setbox0=\hbox\bgroup\strut\hskip\codemargin
  \ifnumlines\lnum\fi}
\def\NL{\egroup\penalty100
  \hbox to\pagewidth{\unhbox0\unskip\hfil}}
\def\RC{\quad\egroup
  \ifdim\wd0<\rcfrac\hsize \wd0=\rcfrac\hsize \fi
  \setbox0=\hbox\bgroup\box0\lcomment\ignorespaces}
\def\BC{\lcomment} % begin comment
\def\EC{\unskip\rcomment} % end comment
% \def\EC{\unskip\rcomment\hskip\rcommentskip\null}
\def\YC#1{$/*$ \ignorespaces #1\unskip\ $*/$} % yacc comment
\def\IN#1{{\dimen0=\slab \multiply\dimen0 by#1 \hskip\dimen0 }} % indent
\def\BL{\penalty0\medskip}
\def\NP{\vskip-\lastskip\vskip0pt plus0.3\pageheight\penalty-200
  \vskip0pt plus-0.3\pageheight\bigskip}
\def\BP{\ifstar\@BP\@@BP}
\def\@BP{\@@@BP\lcommentstar\ignorespaces}
\def\@@BP{\@@@BP\lcomment}
\def\@@@BP{\begingroup \parskip=0.5\baselineskip % begin para comment
  \parindent=0pt \leftskip=\codemargin \advance\leftskip by1.5em
  {\parskip=0pt \parindent=-1.5em \indent}}
\def\EP\NL{\unskip\rcomment\par\endgroup\nobreak}
\def\CQ{\hskip1.5em}
\def\VB{\hbox to1em{$\mid$\hfil}}

\def\K#1{\hbox{\mbf \lowercase{#1}}} % keyword in bold face
\def\I#1{\hbox{\mti #1\/}} % identifier in text italic
\def\S#1{\hbox{\chardef\\=`\\ \chardef\~=`\~ \vsf #1}} % string in sans
\def\R#1{\hbox{\mrm #1}} % odds and ends in roman
\def\C#1{\hbox{\sci \@cap#1\@cap\/}} % contant in caps and small caps

% More stuff for Oberon
\def\D#1{\hbox{\sci \lowercase{#1}\/}} % word in small caps only
\def\O#1{\mathbin{\bf \lowercase{#1}}} % keyword as binary operator
\def\P#1{\unskip\mathrel{\bf \lowercase{#1}}} % keyword as binary relation
	% The \unskip is needed for e.g. ARRAY OF CHAR
\def\HX#1#2{{\mrm #1}_{#2}} % Hex constant

% Hacks for emphasis
\def\U#1\U{\underline{\smash{#1}}}
\def\G#1\G{\hgrey{$#1$}}

% \def\@cap#1#2\@cap{#1\lowercase{#2}} % it's in jmsdef

\def\ANY{\vbox{\hrule width.5em}}
\def\CONS{\mathbin{::}}
\def\AND{\mathrel{\&\&}}
\def\OR{\parallel}
\def\CAT{\mathbin{\raise0.5ex\hbox{$\frown$}}}

% These ones are for Oberon
\def\obAND{\mathrel{\rm \&}}
\def\obOR{\mathrel{\bf or}}
\def\obPTR{\mathord{\uparrow}}
\def\obNOT{\mathord{\sim}}
\def\cat{\mathbin{+\!+}}
\def\dotdot{\mathinner{\ldotp\ldotp}}

\def\LA{\langle}
\def\RA{\rangle}

\def\<{\LA\hbox\bgroup\mrm}
\def\>{\egroup\RA}

\mathchardef\lt=\mathcode`\<
\mathchardef\gt=\mathcode`\>

% A hack to display lines of ML as an alignment
\def\displayml\mlprog#1\endml{\display
  \def\LI##1\NL{\hskip\blockindent##1\cr}%
  \halign to\hsize{##\hfil\tabskip=0pt plus1fil\cr#1}%
  \enddisplay}


%
% Bullets and numbered lists
%

% Nested lists as floats are broken

\newif\ifnewlist
\newskip\normalparskip \normalparskip=0pt

\def\startlist{\ifinfloat\else\par\medspace\fi 
  \begingroup \listsetup}
\def\listsetup{\advance\leftskip by\blockindent \newlisttrue}
\def\listskip{\par \resetpar \ifnewlist\else\medskip\fi \newlistfalse}
\def\listitem#1{\listskip {\parskip=0pt \noindent}%
  \hyperanchor@\itemtag{item}\llap{#1}\ignorespaces}
\def\endlist{\par\endgroup\ifinfloat\else\followpar\fi}
\def\followpar{\global\parskip=\medskipamount
  \global\everypar={\resetpar\hskip-\parindent}}
\def\resetpar{\global\parskip=\normalparskip
  \global\everypar={}\global\nobreakfalse}

\def\enumerate{\startlist\let\item=\enumitem}
\def\enumitem#1{\listitem{\enumlabel{#1}}}
\def\enumlabel#1{\hbox to\blockindent{(#1)\hfil}}
\let\endenum=\endlist

\newcount\alphcount
\def\alphenum{\alphcount=0 \startlist\let\item=\alphitem}
\def\alphitem{\advance\alphcount by1
  \listitem{\enumlabel{\alph\alphcount}}%
  \xdef\lastlabel{{(\alph\alphcount)}{\itemtag}}\ignorespaces}
\def\alph#1{\ifcase#1\or a\or b\or c\or d\or e\or f\or g\or h\or i\or
  j\or k\or l\or m\or n\or o\or p\or q\or r\or s\or t\or u\or v\or
  w\or x\or y\or z\else ?\fi}

\newcount\romcount
\def\romenum{\romcount=0 \startlist\let\item=\romitem}
\def\romitem{\advance\romcount by1
  \listitem{\enumlabel{\romannumeral\romcount}}
  \xdef\lastlabel{{(\romannumeral\romcount)}{\itemtag}}\ignorespaces}

\newcount\arabcount
\def\arabenum{\arabcount=0 \startlist\let\item=\arabitem}
\def\arabitem{\advance\arabcount by1
  \listitem{\enumlabel{\number\arabcount}}
  \xdef\lastlabel{{(\number\arabcount)}{\itemtag}}\ignorespaces}

\def\itemize{\startlist\let\item=\bullitem}
\def\bullitem{\listitem{$\bullet$\enspace}}
\let\enditems=\endlist

\def\description{\par\medspace
  \begingroup\advance\leftskip by2\blockindent
  \let\item=\descitem}
\def\descitem#1{\listskip\noindent
  \hbox to0pt{\kern-\leftskip\bf #1 \ \hfil}\ignorespaces}
\let\enddescription=\endlist

\def\spec{\startlist\let\item=\specitem}
\def\specitem#1{\listskip
  \noindent\hskip-\blockindent
  {\def\\{\unskip\hfil\break}\sf\frenchspacing #1}\par
  \nobreak
  \listitem{$\bullet$\enspace}}
\let\endspec=\endlist

\def\quotation{\startlist\parindent=0pt\listskip{\parskip=0pt \noindent}}
\let\endquote=\endlist

\let\normallistitem=\listitem

\def\meltitem#1{\newlistfalse \let\listitem=\normallistitem
  \hangindent=\leftskip \leftskip=0pt
  \def\par{\endgraf \leftskip=\blockindent \global\let\par=\endgraf}%
  \hyperanchor@\itemtag{item}%
  #1\ignorespaces}

% Stolen from the TeXbook, page 376
\def\futurenonspacelet#1{\def\cs{#1}%
  \afterassignment\stepone\let\nexttoken= }
\def\\{\let\stoken= } \\ % now \stoken is a space token
\def\stepone{\expandafter\futurelet\cs\steptwo}
\def\steptwo{\expandafter\ifx\cs\stoken \let\dummy=\stepthree
  \else \let\dummy=\nexttoken\fi \dummy}
\def\stepthree{\afterassignment\stepone\let\dummy= }

\def\ifstar#1#2{\def\thenpt*{#1}\def\elsept{#2}%
  \futurenonspacelet\next\ifstar@}
\def\ifstar@{\if\next*\let\next=\thenpt\else\let\next=\elsept\fi\next}

\def\ifopt#1#2{\def\thenpt{#1}\def\elsept{#2}%
  \futurelet\next\ifopt@}
\def\ifopt@{\if\next[\let\next=\thenpt\else\let\next=\elsept\fi\next}

\def\paragraphs{\par}
\def\para#1.{\par\resetpar\bigspace\noindent{\it #1: \ }\ignorespaces}
\def\endparas{\par\followpar}

\let\item=\undefined % Don't use plain TeX's \item by mistake


%
% Cross-references
%

% Flag to allow multiple definitions
\newif\ifmultidef \multideffalse

\newread\testin
\def\ifreadable#1{\openin\testin=#1
  \ifeof\testin \closein\testin \immediate\write16{No file #1}%
  \else \closein\testin}
\def\maybeinput#1{\ifreadable{#1} \expandafter\input#1\relax\fi}

% In case of multiple definitions, we keep the first one.
% That gives the aux file priority over other xrefs.
\def\labeldef#1#2#3#4{% label secnum anchor page
  \expandafter\ifx\csname x@#1\endcsname\relax
    \expandafter\def\csname x@#1\endcsname{{#2}{#3}{#4}}
  \else\ifmultidef\else \errmessage{Multiply defined label `#1'}\fi\fi}

\newwrite\auxout
\newif\ifauxopen \auxopenfalse
\def\checkaux{\ifauxopen\else
  \immediate\openout\auxout=\jobname.aux\global\auxopentrue\fi}

\def\label#1{\checkaux
  {\let\thepageno=\relax 
    \edef\next{\write\auxout{\string\labeldef{#1}\lastlabel{\thepageno}}}%
    \next \ifnobreak\ifvmode\nobreak\fi\fi}}

\def\car{\expandafter\@car}
\def\cdr{\expandafter\@cdr}
\def\@car#1#2{#1}
\def\@cdr#1#2{#2}

% These formats for xrefs get args: number anchor page prefix
\def\@@ref#1#2#3#4{\def\next{#2}%
  \ifx\next\empty #4#1\else\hyperlink{#2}{#4#1}\fi}
\def\@@refstar#1#2#3#4{#4#1}
\def\@@page#1#2#3#4{#4#3}

\def\@ref#1#2#3{% macro prefix label
  \leavevmode
  \expandafter\let\expandafter\next\csname x@#3\endcsname
  \ifx\next\relax 
    \immediate\write16{Undefined label `#3'}%
    #2[#3]%
  \else 
    \expandafter#1\next{#2}%
  \fi}

% Use \ref*{label} to suppress the hyperlink
\def\ref{\ifstar{\@ref\@@refstar{}}{\@ref\@@ref{}}}
\def\longref#1{\@ref\@@ref{\hbox{#1}~}}
\def\pageref{\@ref\@@page{}}
\def\refpage{\@ref\@@page{page~}}

\def\lastlabel{{}{}}


%
% Bibliography
%

\newcount\bibcount
\def\biblio{\bibcount=0 \startlist\let\item=\bibitem}
\def\bibitem#1{\advance\bibcount by1
  \listitem{\hbox to\blockindent{\hyperanchor{cite.\number\bibcount}%
    [\number\bibcount]\hfil}}%
  \checkaux
  {\edef\next{\write\auxout{\string\bibdef{#1}{\number\bibcount}%
      {cite.\number\bibcount}}}%
    \next}\ignorespaces}
\let\endbib=\endlist

\def\bibdef#1#2#3{% label number anchor
  \expandafter\ifx\csname b@#1\endcsname\relax
    \expandafter\def\csname b@#1\endcsname{{#2}{#3}}
  \else \errmessage{Multiply defined citation `#1'}\fi}

\def\cite#1{\leavevmode [\@citei#1,,\@end]}
\def\@citei#1,#2\@end{\@citeone{#1}\@citeii#2\@end}
\def\@citeii#1,#2\@end{\def\next{#1}%
  \ifx\next\empty\else ,\penalty1000\ \@citeone{#1}\@citeii#2\@end\fi}

\def\@citeone#1{%
  \expandafter\let\expandafter\next\csname b@#1\endcsname
  \ifx\next\relax
    \immediate\write16{Undefined citation `#1'}%
    #1%
  \else
    \expandafter\@cite\next
  \fi}

\def\@cite#1#2{% number anchor
  \@@ref{#1}{#2}{}{}}



%
% Index entries
%

\newwrite\idxout
\newif\ifidxopen \idxopenfalse
\def\checkidx{\ifidxopen\else
  \immediate\openout\idxout=\jobname.idx\global\idxopentrue\fi}

\def\margnote#1{\insert\margin{\hsize=75pt \baselineskip=8pt
  \parskip=0pt \parindent=0pt \hangindent=8pt \leftskip=0pt
  \eighttt \rightskip=0pt plus1fil \indent\margstrut#1\margstrut\par}}
\def\margstrut{\vrule height5.6pt depth2.4pt width0pt\relax}

\def\specialhat{\ifmmode\def\next{^}\else\let\next=\indexi\fi\next}
\def\indexi{\futurelet\next\indexii}
\def\indexii{\ifx\next\specialhat\let\next=\specialindex
  \else\let\next=\normalindex\fi \next}
\catcode`\^=\active \let^=\specialhat

\def\specialindex^{\begingroup\uncatcodespecials\catcode`\ =10 \specind}
{\catcode`\^=12 \global\def\specind#1^{\endgroup\index{#1}}}

\def\normalindex#1^{\index{#1}#1}

\def\grungearrow{\discretionary{}{>}{>}}

\def\grunge#1>#2>#3\grunge{#1\if!#2!\else\grungearrow#2\fi}

\newif\ifindexnotes \indexnotestrue

\def\index#1{\ifindexnotes\margnote{\grunge#1>>\grunge}\fi
  \checkidx
  {\let\thepageno=\relax \write\idxout{#1&\thepageno}}}


%
% Table of contents
%

% The output file \tocname is opened only when the table of contents
% is printed.  This has the bad effect that headings before the table
% of contents will not appear -- and the good effect that the table of
% contents does not include itself, something wise printers have
% always felt to be silly.

\edef\tocname{\jobname.toc}

\newwrite\tocout
\newif\iffirsttoc

\def\thecontents{\firsttoctrue \maybeinput{\jobname.toc}
  \immediate\openout\tocout=\tocname
  \immediate\write\tocout{\relax}
  \let\addtoc=\doaddtoc}

\def\doaddtoc#1#2#3#4#5{% macro number title page bmark
  {\let\thepageno=\relax \enableprotectii
    \edef\next{\write\tocout{\string#1{#2}{#3}{#4}{#5}}}
    \next}}

% Protect for two expansions:
\def\enableprotectii{\def\protect{\noexpand\noexpand\noexpand}}
% Protect for multiple expansions:
\def\enableprotect{\def\protect{\noexpand\protect\noexpand}}

\def\addtoc#1#2#3#4#5{}

\newdimen\tocfudge
\setbox0=\hbox{\elevensfb 0}\tocfudge=\wd0
\setbox0=\hbox{\tenrm 0}\advance\tocfudge by-\wd0

\def\tocinsert#1#2#3#4{\iffirsttoc\firsttocfalse\else\bigskip\fi
  \line{\elevensfb\def\pit{\elevensfi}\hbox to2\fontquad{#1\hfil}#2\hfil #3}}
\def\toclinei#1#2#3#4{\tocinsert{#1}{#2}{\hyperlink{#4}{#3}}{}}
\def\toclineii#1#2#3#4{\line{\tenrm\hskip\blockindent
    \hbox to3\fontquad{#1\hfil}#2\hfil
    \hyperlink{#4}{#3}\kern0.5\tocfudge}}


%
% Index and cross reference listing
%

\def\beginindex{\begindoublecolumns \eightpoint
  \parskip=0pt plus1pt \rightskip=0pt plus5em
  \exhyphenpenalty=10000} % don't break 136--7
\def\indexitem{\par\hang\noindent}
\def\indexspace{\par\bigspace}
\def\endindex{\par\enddoublecolumns}

\def\UU#1{$\underline{\hbox{#1}}$}


%
% Changed pages
%

\newwrite\chgout
\newif\ifchgopen \chgopenfalse
\def\checkchg{\ifchgopen\else
  \immediate\openout\chgout=\jobname.chg\global\chgopentrue\fi}

\def\changepage{\afterassignment\@chgpg\count@=}
\def\@chgpg{\checkchg\immediate\write\chgout{\the\count@}}

\def\change{\checkchg\write\chgout{\the\pageno}}


%
% Hooks for hypertext
%

\def\hyperlinks{\input jmshyp }

\def\hyperlink#1#2{#2}
\def\hyperanchor#1{}
\def\hyperanchor@#1#2{\def#1{}}
\def\bookmark#1#2#3{}
\def\bookmark@#1#2#3{}

\def\bmark{}

% Switch to make hyperlinks for exercise numbers (appropriate only if
% answers are included in the same document)
\newif\ifhyperex \hyperexfalse
\def\hyperex{\hyperextrue}

%
% Bits and pieces
%

% Read in the aux file if it exists
\maybeinput{\jobname.aux}

\let\protect=\relax

\def\ocaml{Objective {\sc Caml}}

% Stolen from LaTeX's lfonts.tex
\def\em{\protect\pem}
\def\pem{\ifdim \fontdimen1\font>0pt \rm \else \it \fi}

\def\today{\number\day/\number\month/\number\year}

\def\hbreak{\hfil\break}

\def\fn#1{\hbox{\def\\{\char`\\}\vsf #1}}
\def\pn#1{\hbox{\it #1\/}}

{\catcode`\^=7 \global\catcode`\^^I=15} % TAB characters are illegal

\let\normalbaselineskip=\undefined

\showboxbreadth=1000
\showboxdepth=20

\normalpage

\catcode`@=12

%%%%% Special stuff

\def\unix{{\sc unix}}
\def\Unix{{\sc Unix}}

% \point marks an important point with a marginal arrow
\def\point{\strut
  \vadjust{\vbox to0pt{\vss\line{\strut\hfil\rlap{\enspace$\triangleleft$}}%
     \vskip0pt}}}

\def\Question{\point {\it Question:\/} }

% We use fixed-width typewriter for verbatim text
\def\verbfont{\tt} \def\verbifont{\tti}
\def\verbatim{\beforedisplay\begingroup
  \setupverbatim\let\verbsp=\ \doverbatim}

% Let's be pragmatic: some verbatim displays are just too wide
\def\wideverb{\beforedisplay\begingroup\advance\hsize by70pt
  \setupverbatim\let\verbsp=\ \doverbatim}
\def\wideinter{\beforedisplay\begingroup\advance\hsize by70pt
  \setupverbatim\catcode`\_=\active\italverb\dointeract}

\def\program{\beforedisplay\begingroup
  \setupverbatim\setupcomment\progfont\doprogram}
\def\setupcomment{\catcode`\%=\active \comment}
{\catcode`\%=\active \global\def\comment{\let%=\halfcomm}}
{\catcode`\^=7 \catcode`\^^M=\active%
  \global\def\halfcomm#1^^M{\hfill\hbox to0.5\hsize{#1\hfil}\par}}

%%%%% End of specials
